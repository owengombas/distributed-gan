\chapter{Open Science principles}

\section{Open-Source Tools}
Our project used open-source software, ensuring that all code and development environments are accessible to the research community and beyond. We developed our Distributed-GAN framework using Python and PyTorch, both of which are open-source, to facilitate ease of use and adaptation by other researchers. 

The entire implementation of our GAN model, including the architecture and algorithms, is shared on GitHub\footnote{Code available on \url{https://github.com/darmangerd/distributed-gan}}, promoting collaboration and further development within the community.


\section{Datasets}
We used publicly available datasets, such as MNIST and CIFAR-10, to train our distributed GANs. These datasets are widely recognized and provide a benchmark for evaluating our model's performance against established standards. By detailing our data use and experimental setups transparently in our documentation, we ensure that other researchers can replicate our studies or extend them with new data under similar conditions.


\section{Reproducibility of Results}
A key aspect of our project's alignment with open science is our commitment to reproducibility. We have documented all experimental procedures, model configurations, and hyperparameter settings. This comprehensive documentation ensures that others can reproduce our results and verify our claims. Additionally, by reproducing the results from the foundational MD-GAN paper, we contribute to the validation of previous findings within the community.